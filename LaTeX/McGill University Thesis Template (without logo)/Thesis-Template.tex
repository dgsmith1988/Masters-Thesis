% Authors: Simon Geoffroy-Gagnon and Farhad Shokraneh
% Based on the Thesis by Rubana Bahar Priti
% Edit: 2020.04.16

\documentclass[12pt, TexShade, letterpaper]{report}
\usepackage[utf8]{inputenc}
\usepackage{palatino}
\usepackage{amsmath}
\usepackage{amssymb}
\usepackage{graphicx}
\usepackage[labelfont=bf]{caption}
\usepackage{subcaption}
\usepackage{setspace}
\captionsetup[table]{font = {stretch=1.35}}
\captionsetup[figure]{font = {stretch=1.35}}
\usepackage[margin=1in, headsep=1cm, bottom=5cm]{geometry}
\usepackage[hidelinks]{hyperref}
\newcommand{\MYhref}[3][blue]{\href{#2}{\color{#1}{#3}}}%
\usepackage{tabu}
\usepackage{cite}
\usepackage[table]{xcolor}
\usepackage{nomencl}
\usepackage[nonumberlist,nogroupskip,xindy]{glossaries}
\usepackage{floatrow}
\usepackage{wrapfig}
\renewcommand{\baselinestretch}{2}
\usepackage{fancyhdr}
\usepackage{lmodern}

\renewcommand{\chaptermark}[1]{\markboth{#1}{}} % Ensure List of Figs, ToC, and glossary are named in the header

\setglossarystyle{long}
\renewcommand{\glsnamefont}[1]{\textbf{#1}}

\newglossarystyle{mystyle}{%
	\renewenvironment{theglossary}%
	{\begin{longtabu} to \linewidth {p{0.15\linewidth}p{0.85\linewidth}}}%
		{\end{longtabu}}%
	\renewcommand*{\glossaryheader}{}%
	% indicate what to do at the start of each logical group
	\renewcommand*{\glsgroupheading}{}
	\renewcommand*{\glsgroupskip}{}%
	\renewcommand*{\glossaryentryfield}[5]{%
		\glstarget{##1}{##2}% Name
		& ##3% Description
		\\% end of row
	}
}

% Overwrite the plain page style with a red line and page numbering
\fancypagestyle{plain}{%
	\fancyhf{} % clear all header and footer fields
	\fancyhead[R]{\textbf{\thepage}} % except the center
}

% Create the fancy page style header
\pagestyle{fancy}
\fancyhf{}
\lhead{\textbf{\nouppercase{\leftmark}}}
\chead{}
\rhead{\textbf{\thepage}}


\usepackage{xpatch}
\xpretocmd\headrule{\color{red}}{}{\PatchFailed}

% Get rid of all dashed words
\tolerance=1
\emergencystretch=\maxdimen
\hyphenpenalty=10000
\hbadness=10000

% Create glossary
\newacronym{cmos}{CMOS}{complementary metal–oxide–semiconductor}
\newacronym{nn}{NN}{neural network}
\newacronym{onn}{ONN}{optical neural network}
\newacronym{gpu}{GPU}{graphical processing unit}
\newacronym{ml}{ML}{machine learning}
\newacronym{relu}{ReLU}{rectified linear unit}
\newacronym{mse}{MSE}{mean squared error}
\newacronym{ps}{PS}{phase shifter}
\newacronym{dc}{DC}{directional coupler}
\newacronym{mzi}{MZI}{Mach-Zehnder interferometer}
\newacronym{il}{IL}{insertion loss}
\newacronym{flop}{FLOP}{floating point operation}
\newacronym{flops}{FLOPS}{floating point operation per second}
\newacronym{pca}{PCA}{principle component analysis}
\newacronym{api}{API}{application programming interface}

\newacronym{svd}{SVD}{singular value decomposition}
\newacronym{dmm}{DMM}{Diagonal Matrix Multiplication}
\newacronym{omm}{OMM}{optical matrix multiplication}
\newacronym[firstplural=figures of merit (FoMs)]{fom}{FoM}{figure of merit}
\newacronym{mnist}{MNIST}{modified national institute of standards and technology}


\makeindex
\makeglossaries

% Set page numbering to roman
\setcounter{page}{2}\renewcommand{\thepage}{\roman{page}}

\author{\textcopyright Author, August, 2020}
\date{}

\begin{document}

\begin{titlepage}
		\begin{center}
			\vspace*{0.5cm}

			\LARGE
			\textbf{Title}
			
			\vspace{1cm}
			
			\textit{Author}
			
			\vspace{7cm}
			
			% \includegraphics[width=0.25\textwidth]{mcglogo.png}
			
			\Large
			Department of Electrical \& Computer Engineering
			
			\vspace{-5mm}
			McGill University
			
			%\vspace{-5mm}
		%	Montr\'eal, Qu\'ebec, Canada
			
			\vspace{5mm}
			August 15, 2020
			\small
			\vspace{0.5cm}
			{\color{red} \hrule height 0.75mm}
			
			\vspace{0.3cm}
			A thesis submitted to McGill University in partial fulfillment of the requirements of the degree of Electrical Engineering
			
			\copyright\hspace{0.5mm}2020 Author
			
		\end{center}
	\end{titlepage}
\setlength{\voffset}{2cm}
\renewcommand{\chaptermark}[1]{%
	\markboth{\thechapter.\ #1}{}}
\chapter*{Abstract}\markboth{Abstract}{}
	\label{chap:engAbstract}
%	\addcontentsline{toc}{section}{\nameref{chap:engAbstract}}

\chapter*{Abrégé}\markboth{Abrégé}{}
	\label{chap:frAbstract}
%	\addcontentsline{toc}{section}{\nameref{chap:frAbstract}}

\chapter*{Acknowledgements}\markboth{Acknowledgements}{}
	\label{chap:acknowledgments}
%	\addcontentsline{toc}{section}{\nameref{chap:acknowledgments}}
	
 % Start of ToC, LoT, gls
	\tableofcontents\thispagestyle{plain}

	\listoffigures\thispagestyle{plain}
%	\addcontentsline{toc}{section}{\listfigurename}
	\listoftables
%	\addcontentsline{toc}{section}{\listtablename}
	\glsaddall
	\setlength\LTleft{0pt}
	\setlength\LTright{0pt}
	\setlength\glsdescwidth{0.8\hsize}
	\printglossary[title={List of Acronyms}]
	\markright{List of Acronyms} 

 	\clearpage
	\pagenumbering{arabic} % restart page numbers at one, now in arabic style
	
	\glsresetall
	% start of mainmatter
\chapter{Introduction}
	\section{Thesis objective}
	\section{Thesis organization}
    \section{Research contributions}
    	\subsection{Journal publications}
    	\subsection{Paper contributions}
    	\subsection{Open source code contribution}
	
\chapter{Chapter Two}
\section{Section 1}
\subsection{Subsection 1}
\subsubsection{Subsubsection 1}
\cite{lauer2007trainable}

\chapter{A Third Chapter}

\chapter{Another Chapter}
	
\chapter{Conclusion}

	% Begin Bibliography
	{
	
	\bibliography{references}
	\bibliographystyle{ieeetr}
	
	}
\appendix
\chapter{Equations}
\begin{equation}
    E = M\cdot C^2
\end{equation}

\chapter{Tables}

\end{document}