%!TEX root = main.tex

\documentclass[../main.tex]{subfiles}

\begin{document}

\chapter{Conclusion and Future Work}

\section{Conclusion}
This thesis served the purpose of developing a physically informed synthesis model for the slide guitar. The model is based on the design described in \citetwo{pakarinen_virtual_2008} and implemented in \citetwo{puputti_real-time_2010}. These papers also contain an emphasis on the gestural control aspects and comparatively some of the digital signal processing aspects are neglected.  The thesis has a secondary goal of extending and elucidating the digital signal processing techniques related to the implementation as the previously mentioned papers were focused on other topics.

First, the slide guitar was described, emphasizing how it differs from the traditional fretted playing style in that the slide acts as a moving string termination. This allows for a continuum of pitches and nuanced articulations to be produced (including vibrato). The basics of digital waveguide modeling and acoustics were introduced to be able to explain how they can be used to create a computational model of a transversely vibrating string. This model was then adapted to support emulation of the slide guitar. An overview to the handful of more recent approaches related to slide guitar sound synthesis was then described, including comparisons between the different approaches.

After the basics were introduced in Chapter~\ref{ch:background}, a more detailed explanation of the slide guitar model's architecture was explained in Chapter~\ref{ch:three}. This included a digital waveguide model which allows for a time-varying pitch and more precise tuning via delay-line interpolation. Also included is a Contact Sound Generator to emulate the interactions between the slide and string surfaces. The explanation of the model included a discussion of the limitations inherent in its design. Modifications to the inspiration model include a gain coefficient to control the longitudinal-to-transverse coupling as well as the addition of the Noise Burst Generator and Harmonic Resonator Bank as potential components to increase the creative potential of the synthesizer. The full details of implementation for each component as well as audio examples are provided to develop a more intuitive understanding of each component. This is an extension to the initial model descriptions.

Subsequently in Chapter~\ref{ch:four}, the verification techniques used to ensure the correct operation of the model and its constituent components were discussed and tested. The results of Chapter~\ref{ch:four} illustrated that the various constituent components (e.g. filters, delay lines, string digital waveguides, etc.) worked correctly as well as the single string slide synthesizer as a whole. This is a substantial improvement over the results presented in \citetwo{puputti_real-time_2010} as there is comparatively little objective evidence indicating the implementation works correctly.

A discussion then followed in Chapter~\ref{ch:5} regarding the physical measurements performed to help verify the model's accuracy as well as refine its implementation. Experiments which prove the existence of the longitudinal-to-transverse coupling as well as attempt to measure $T_{60}$ for the wound strings are included. The sound design and tuning aspects which these measurements informed were then discussed in addition to some subtle implementation details. This information is not covered in \citetwo{pakarinen_virtual_2008} or \citetwo{puputti_real-time_2010}. There is a strong physical foundation which informs the design of the synthesizer.

In Chapter~\ref{ch:Ch6}, aspects related to the parametrization of the control signal and synthesis parameters are discussed. The best method for configuring the Contact Sound Generator to generate realistic sounds is described. The control signal is examined in terms of the qualities which generate realistic playing examples as well as how to purposefully induce an unstable state in the synthesis model for creative purposes. There is also a comparison between the synthesized and recorded results of the same musical example. Minimal discussion of these topics occurs in \citetwo{pakarinen_virtual_2008} and \citetwo{puputti_real-time_2010}. In summary, the model itself produces usable results with similar spectral characteristics to recordings but lacks the nuances associated with playing a physical instrument.

In summary, this thesis implements and expands on the model introduced and implemented in the aforementioned papers. More precise implementation details are described and limitations of the model are examined. Physical measurements have been completed to refine and verify the model's accuracy. Guiding principles behind configuration of the parameters and control signals for sound design are discussed. Auditory examples have been provided throughout to help illustrate the different concepts.

\section{Future Work}
There remain many opportunities for improvement of the model. In terms of improving its constituent components the impulse train generator could be modified to a Band-Limited Impulse Train. This would support the generation of impulse signals which are not quantized based on an integer number of sample periods and overcome a current limitation. The Wound CSG could be modified to change the harmonic strengths based on the slide speed to more accurately model the measurements made on the physical system.

In terms of the loop filter, two obvious refinements could be made. Given that the phase delay values at different string lengths are not constant, this could be made more accurate by using the mean of the phase delay across the spectrum instead of a constant approximation. To help reduce the computational complexity and facilitate a real-time implementation, a look-up table based on $L[n]$ could be developed for each string based on these average values. The loop filter could also be redesigned so that the magnitude response always stays below unity beyond the 19th fret. This would extend the range of usable $L[n]$ values to incorporate situations which are likely occur during slide playing. Another approach could be investigating the bridge and body of the instrument to account for the coupling between strings. This could be incorporated into a modeling filter which could help provide a more realistic sound.

Many different aspects of the physical measurements could also be refined. The coupling and spectral differences between fresh and older corroded strings could be measured and incorporated into the model to provide more timbral options. An experimental design to capture the contact sound generated by the unwound strings could be developed given the muting issues which occurred. The setup used to measured the $T_{60}$ for each string has a lot of opportunity for improvement and a more accurate $T_{60}$ would help illustrate deficiencies in the Wound CSG model. Also, the longitudinal-to-transverse measurements could be made more quantitative and these could be used to help refine the coupling model. A filter, as opposed to a scaling coefficient, would likely be a better candidate given the coupling is likely influenced by the impedance of the string termination at the bridge. All the experiments could benefit from a physical setup where the guitar was guaranteed to be decoupled from its supporting structure (or the coupling was quantified).
\end{document}