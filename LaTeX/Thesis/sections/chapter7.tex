%!TEX root = main.tex

\documentclass[../main.tex]{subfiles}

\begin{document}

\chapter{Conclusion and Future Work}

\section{Conclusion}
This thesis served the purpose of developing a physically informed synthesis model for the slide guitar. First, the slide guitar as a concept was introduced, emphasizing how it differs from the traditional fretted playing style in that the slide acts as a moving string termination. This allows for a continuum of pitches and nuanced articulations to be produced. The basics of digital waveguide modeling and acoustics were introduced to be able to explain how they can be used to create a computational model of a transversely vibrating string. This model was then adapted to support emulation of the slide guitar. An overview to the handful of more recent approaches related to slide guitar sound synthesis was then described, including comparisons between the different approaches.

After the these basics were introduced, a more detailed explanation of the slide guitar model's architecture was explained. This includes a digital waveguide model which allows for a time-varying pitch and more precise tuning via delay-line interpolation. Also included is a Contact Sound Generator to emulate the interactions between the slide and string surfaces. The explanation of the model included a discussions of the limitations inherent in its design. Subsequently, the verification techniques used to ensure the correct operation of the model and its constituent components were discussed.

A discussion then followed regarding the physical measurements performed to help verify the model's accuracy as well as refine its implementation. The sound design and tuning aspects which these measurements informed were then discussed as well some subtle implementation details. Aspects related to the parametrization of the control signal are discussed in terms of generating realistic playing examples as well as purposefully inducing an unstable state in the synthesis model for creative purposes.

\section{Future Work}
There remain many opportunities for improvement of the model. In terms of improving its constituent components the impulse train generator could be modified to a Band-Limited Impulse Train. This would support the generation of impulse signals which are not quantized based on an integer number of sample periods and overcome a current limitation. The Wound CSG could be modified to change the harmonic strengths based on the slide speed to more accurately model the measurements made on the physical system.

In terms of the loop filter, two obvious refinements could be made. Given that the phase delay values at different string lengths are not constant, this could be made more accurate by using the mean of the phase delay across the spectrum instead of a constant approximation. To help reduce the computational complexity and facilitate in a real-time implementation, a look-up table based on $L[n]$ could be developed for each string based on these average values. The loop filter could also be redesigned so that the magnitude response always stays below unity beyond the 19th fret. This would extend the range of usable $L[n]$ values to incorporate situations which are likely occur during slide playing.

Many different aspects of the physical measurements could also be refined. The coupling and spectral differences between fresh and older corroded strings could be measured and incorporated into the model to provide more timbral options. An experimental design to capture the contact sound generated by the unwound strings could be developed given the muting issues which occurred. The setup used to measured the $T_{60}$ for each string has a lot of opportunity for improvement and a more accurate $T_{60}$ would help illustrate deficiencies in the Wound CSG model. Also, the longitudinal-to-transverse measurements could be made more quantitative and these could be used to help refine the coupling model. A filter, as opposed to a scaling coefficient, would likely be a better candidate given the coupling is likely influenced by the impedance of the string termination at the bridge.
\end{document}