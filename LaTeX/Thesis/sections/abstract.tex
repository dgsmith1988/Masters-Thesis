%!TEX root = main.tex

\documentclass[../main.tex]{subfiles}

\begin{document}

\chapter*{Abstract / Résumé}
\section*{English}
This thesis introduces a digital waveguide model for simulating the sound of a slide guitar. This model uses a single digital waveguide model for the transverse vibrations of the string as well as an additional module which models the contact sound between the slide and string surfaces. After introducing the background and previous inspiration model, a more well-developed architecture is defined and introduced. Included in this is an examination of its limitations. This model was implemented in MATLAB and facilitates sound synthesis on a sample-by-sample computational basis. The verification techniques used in the developed model are described. As this is a physical model, various measurements were made on an actual acoustic guitar in an attempt to verify the model's accuracy as well as refine aspects missing from previous attempts. After these measurements are made, the strategies used to parameterize the model for sound design and musicality are discussed.

\section*{Français}
Ce mémoire présente un modèle de guide d’ondes numérique pour simuler le son d’une guitare slide. Le modèle utilise un seul guide d’ondes numérique pour les vibrations transversales de la corde avec un module additionnel qui modèle le son du contact entre la slide et la surface des cordes. Après une introduction qui présente le contexte et le modèle ayant servi d’inspiration, une architecture plus avancée est présentée. Il s'agit notamment d’examiner ses limites. Le modèle a été mis en œuvre sur MATLAB et faciltie la synthèse sonore sur la base de calculs échantillon par échantillon. Les techniques de vérification appliquées au modèle développé sont décrites. Comme il  s’agit d’un modèle physique, des mesures ont été effectuées sur une guitare acoustique pour vérifier la précision du modèle et affiner les aspects que les tentatives précédentes auraient négligés. Après avoir traité de ces mesures, le mémoire présente les stratégies de paramétrage du modèle pour la conception sonore et la musicalité.
\end{document}