%!TEX root = main.tex

\documentclass[../main.tex]{subfiles}

\begin{document}

\chapter{Introduction}
Fill me in later

% \section{Motivation}
% Fill me in later

% \section{Objectives}
% Fill me in later

% \section{Thesis Organization}
% The thesis is organized into six different chapters as described below. TODO: REWRITE THIS AFTER FINISHING THE OTHER PARTS.

% \subsubsection{Chapter 1 - Introduction}
% This section provides a brief overview of what the thesis covers as well as the reasons the thesis was undertaken and what it hopes to achieve.

% \subsubsection{Chapter 2 - Background}
% This section elaborates upon the necessary theoretical as well as musical knowledge to understand the rest of the thesis. It is not meant to be exhaustive as both digital signal processing and slide guitar are large and broad fields themselves. The aim is to introduce as much theory as is necessary as well as provide resources if an interested reader prefers to get more details. Other approaches and techniques will be examined as well.

% \subsubsection{Chapter 3 - Description of Model}
% In this section the architecture of the synthesizer as well as its constituent components will be described. Some aspects will have already been introduced in the background, however more details will be provided to provide clarity as the source material didn't not explain all the implementation details. .

% \subsubsection{Chapter 4 - Verification Of Model and Constituent Components}
% The techniques used to verify the correctness of the model will be discussed as well as any limitations inherent to it. Audio examples will be provided to help develop an aural intuition for the sounds, however the synthesis parameters will often not be realistic and physically informed as their goal is to illustrate algorithmic correctness (as opposed to usable sonic potential)

% \subsubsection{Chapter 5 - Physical Measurements}
% In this section we will describe the physical experiments which were performed during the development of this model. As will be shown in both chapters 2 & 3, there is a strong physical basis for the synthesizer model. Many of the model's parameters have a physical correlate, hence why this section comes after the model's description. Some of the experiments will be recreating work from the original paper, while others will attempt to refine the model to make it more physically informed.

% \subsubsection{Chapter 6 - Sound Design}
% In this section we will explore how the parameters were tuned as well as the different architectural and component decisions which were explored in an attempt to create the "best" sounding synthesizer. This has been placed at the end as the physical measurements were useful in evaluating different design decisions. Additionally, an approach to extracting the pitch trajectory of a recorded slide example will be introduced and evaluated to make the resulting sounds more realistic.

% \subsubsection{Chapter 7 - Conclusion}
% This section provides a brief summary of what has been explored in the thesis as well as expounds upon opportunities for improvement and future research.

\end{document}
