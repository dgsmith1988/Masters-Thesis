\documentclass[12pt]{article}
\usepackage{geometry}
\geometry{margin=2cm, top=2cm, bottom=2cm}
\usepackage{amsmath}
\usepackage{titling}

\setlength{\droptitle}{-8em}
\title{M.A. Thesis Proposal: Virtual Slide Guitar \\David ``Graham'' Smith}
%\date{}

\begin{document}
%\maketitle

\begin{flushleft}
    \large \textbf{M.A. Thesis Proposal: Virtual Slide Guitar}
    \hfill
    \normalsize David ``Graham'' Smith
\end{flushleft}
\hrule

\subsection*{Introduction / Motivation}
\paragraph{}
One of the more unique methods of playing guitar is an approach referred to as “slide guitar”. This consists of using a rigid object, frequently a metal or glass tube, to control the length of the string, instead of the frets. The slide acts as a string termination and influences the vibration of the string by creating a new load impedance in between the nut and the bridge. This allows unique articulations and pitch inflections to be generated as the player is no-longer constrained to the pitches provided by the fret-locations. Additionally, the interaction of the slide’s surface with that of the string adds a new timbral component related to the slide’s velocity. 

In the case of wound strings, this adds two new sounds. The first is a time-varying harmonic component due to the interaction of the slide with the spatially periodic pattern of windings on the string’s surface (inherent in a wound string’s construction).  The second component is due to the stimulation of the string’s longitudinal modes as the slide introduces disturbances in this direction when it impacts the ridges of the windings. As the slide does not provide sufficient force to change the longitudinal length, the modal frequencies are static, regardless of the motion of the slide. 

Slides and unwound strings are traditionally made of smooth polished materials, chrome/brass/glass and metal respectively. Correspondingly, the coefficient of friction between the two is comparatively much lower than in the wound-string case. Unwound strings also have a uniform surface, lacking the ridges created by windings which a slide impacts while traveling the length of the string. This darastically reduces the coupling between the slide and the unwound string from a longitudinal standpoint, preventing the longitudinal modes from being audible. As a result, the contact sound generated for unwound strings is more closely modeled by white-noise scaled by the slide velocity. 

The slide-guitar hasn’t been explored nearly as much from a synthesis standpoint compared to the traditional fretted guitar. Part of my motivation comes from this novelty. Additionally, I have long been a guitar player and slide guitar enthusiast, so understanding the physical mechanisms at play in slide articulation (as well as guitar playing in general) are of keen interest to me personally.

\subsection*{Previous Work}
\paragraph{}
Sound synthesis of a vibrating string has long been of interest to researchers as illustrated by the work of Hiller-Ruiz \cite{hiller_synthesizing_1971} \cite{hiller_synthesizing_1971-1}, Karplus-Strong \cite{karplus_digital_1983} and Jaffe-Smith \cite{jaffe_extensions_1983}. Instrument specific extensions were developed by Välimäki et al. \cite{vaelimaeki_physical_1996}.  Karjalainen et al formalized the more computationally efficient Single Delay Loop (SDL) model \cite{karjalainen_plucked-string_1998}. More recently, guitar specific plucking has been researched by James Woodhouse in a pair of papers \cite{woodhouse_synthesis_2004} \cite{woodhouse_plucked_2004}. None of this work focused on the slide guitar. The first published journal article regarding this topic came out of a group of Finnish researchers in 2008 \cite{pakarinen_virtual_2008}. This material is heavily tied to a master’s thesis by Tapio Puputti published in 2010 \cite{puputti_real-time_2012}. This set of papers involves a substantial gestural control component as they aimed to extend a pre-existing Virtual Air Guitar system making the sound synthesis aspect less emphasized. Their approach uses a digital waveguide (DWG) model based on other work from the Helsinki University of Technology.

The same group of researchers has conducted prior work which was incorporated into the slide DWG model. A paper of theirs from 2007 focuses on the interaction noises of fingers with wound strings \cite{pakarinen_analysis_2007}. This heavily informs the model of slide/string surface interaction and translates into a parametric model dependent on the slide and string properties. The same paper influenced a synthesis model of the Chinese guqin \cite{penttinen_model-based_2006}. One of the distinguishing characteristics of the guqin is that it is a fretless plucked instrument which uses wound strings. Correspondingly, the friction model developed for the Virtual Slide Gutiar (VSG) model is heavily influenced by the guqin model's.

Recent work has attempted to extend the DWG model or use a finite-difference approach. Gianpaolo Evangelista has a history of modeling player/instrument interaction aspects of guitar playing \cite{evangelista_physical_2011} \cite{evangelista_playerinstrument_2010}. His latest approach extendeds the Finnish DWG model to make the model more physically informed \cite{evangelista_physical_2012}. Researchers at Queen’s University Belfast have produced a finite difference model approach which produces quality results \cite{bhanuprakash_finite_2020}.

\subsection*{Proposed Research / Methodology}
\paragraph{}
A synthesis model following Puputti's thesis will be developed which supports sample-by-sample calculations of the sounds. The constituent components (i.e. pulse generator, loop-filter, fractional delay...) will be made as modular as possible to allow for an exploration of different approaches to each component and their effect on the sound-synthesis system in general. This includes a SDL acting as the waveguide, a 5th-order Lagrange interpolator as the fractional delay element and zero-order energy-preserving interpolation to implement the energy compensation.

Physical measurements will be made recreating the original approach to capturing the longitudinal modes. This will be done using my MINIDSP UMIK-2 to capture the sound in the semi-anechoic room available at CIRRMT. The instrument used will be my own Yamaha steel string acoustic guitar as well as my own brass/chrome/glass slides. The number of windings on each string and their dimensions will be measured using the micrometer available in the music tech area recreating the analysis for the friction sounds from the guqin paper \cite{penttinen_model-based_2006}. This information is used to help calibrate the contact sound generator part of the model. Other measurements will be made as the project progresses to help extend/verify the model. Ultimately the limitations of the synthesis model will dictate subsequent investigations. Techniques from the more recent approaches, such as in \cite{bhanuprakash_finite_2020} or \cite{evangelista_physical_2012}, will be investigated to assess their contributions to realism or creative potential (ie. improving the loop filter to incorporate the body effects, adding multiple polarizations...). 

After establishing a complete synthesis model, the next step is to determine the effective parameter ranges which produce musical results. The best set of synthesis parameter configurations will be characterized. This could be done in real-time using a GUI, ultimately dependent on the nature of the control signals involved and the best method to control them. Iconic slide licks will serve as the basis of emulation given that the goal is to be able to use the model to create musical sounds.

\subsection*{Contributions / Summary}
\paragraph{}
To summarize, a synthesis model of the slide guitar will be developed. This will be based on the DWG model proposed by the Finnish researchers to start. After establishing correctness of the initial algorithm, explorations will be made to investigate how different parametrizations/implementations of the constituent components affect the sound. Physical measurements will be made to examine aspects such as the winding noise and longitudinal mode characteristics. Once the best model implementation has been decided, the parameter space will be explored to see what produces the most usable and musical results.

\clearpage
\bibliography{references}
\bibliographystyle{alpha}

\end{document}