\documentclass[12pt]{article}
%DIF LATEXDIFF DIFFERENCE FILE
%DIF DEL C:\Users\Graham\Documents\McGill\M.A\Thesis\Masters-Thesis\LaTeX\Proposal\proposal5.tex   Sun Dec 18 12:40:08 2022
%DIF ADD C:\Users\Graham\Documents\McGill\M.A\Thesis\Masters-Thesis\LaTeX\Proposal\proposal6.tex   Fri Jan 13 15:06:15 2023
\usepackage{geometry}
\geometry{margin=2cm, top=2cm, bottom=2cm}
\usepackage{amsmath}
\usepackage{titling}
%DIF 6a6
\renewcommand{\refname}{\normalsize{References}} %DIF >
%DIF -------

\setlength{\droptitle}{-8em}
\title{M.A. Thesis Proposal: Virtual Slide Guitar \\David ``Graham'' Smith}
%\date{}
%DIF PREAMBLE EXTENSION ADDED BY LATEXDIFF
%DIF UNDERLINE PREAMBLE %DIF PREAMBLE
\RequirePackage[normalem]{ulem} %DIF PREAMBLE
\RequirePackage{color}\definecolor{RED}{rgb}{1,0,0}\definecolor{BLUE}{rgb}{0,0,1} %DIF PREAMBLE
\providecommand{\DIFadd}[1]{{\protect\color{blue}\uwave{#1}}} %DIF PREAMBLE
\providecommand{\DIFdel}[1]{{\protect\color{red}\sout{#1}}}                      %DIF PREAMBLE
%DIF SAFE PREAMBLE %DIF PREAMBLE
\providecommand{\DIFaddbegin}{} %DIF PREAMBLE
\providecommand{\DIFaddend}{} %DIF PREAMBLE
\providecommand{\DIFdelbegin}{} %DIF PREAMBLE
\providecommand{\DIFdelend}{} %DIF PREAMBLE
\providecommand{\DIFmodbegin}{} %DIF PREAMBLE
\providecommand{\DIFmodend}{} %DIF PREAMBLE
%DIF FLOATSAFE PREAMBLE %DIF PREAMBLE
\providecommand{\DIFaddFL}[1]{\DIFadd{#1}} %DIF PREAMBLE
\providecommand{\DIFdelFL}[1]{\DIFdel{#1}} %DIF PREAMBLE
\providecommand{\DIFaddbeginFL}{} %DIF PREAMBLE
\providecommand{\DIFaddendFL}{} %DIF PREAMBLE
\providecommand{\DIFdelbeginFL}{} %DIF PREAMBLE
\providecommand{\DIFdelendFL}{} %DIF PREAMBLE
%DIF COLORLISTINGS PREAMBLE %DIF PREAMBLE
\RequirePackage{listings} %DIF PREAMBLE
\RequirePackage{color} %DIF PREAMBLE
\lstdefinelanguage{DIFcode}{ %DIF PREAMBLE
%DIF DIFCODE_UNDERLINE %DIF PREAMBLE
  moredelim=[il][\color{red}\sout]{\%DIF\ <\ }, %DIF PREAMBLE
  moredelim=[il][\color{blue}\uwave]{\%DIF\ >\ } %DIF PREAMBLE
} %DIF PREAMBLE
\lstdefinestyle{DIFverbatimstyle}{ %DIF PREAMBLE
        language=DIFcode, %DIF PREAMBLE
        basicstyle=\ttfamily, %DIF PREAMBLE
        columns=fullflexible, %DIF PREAMBLE
        keepspaces=true %DIF PREAMBLE
} %DIF PREAMBLE
\lstnewenvironment{DIFverbatim}{\lstset{style=DIFverbatimstyle}}{} %DIF PREAMBLE
\lstnewenvironment{DIFverbatim*}{\lstset{style=DIFverbatimstyle,showspaces=true}}{} %DIF PREAMBLE
%DIF END PREAMBLE EXTENSION ADDED BY LATEXDIFF

\begin{document}
%\maketitle

\begin{flushleft}
    \large \textbf{M.A. Thesis Proposal: Virtual Slide Guitar}
    \hfill
    \normalsize David ``Graham'' Smith
\end{flushleft}
\hrule

\subsubsection*{Introduction / Motivation}
\paragraph{}
One of the more unique methods of playing guitar is an approach referred to as “slide guitar”. This consists of using a smooth rigid tube (the slide) to control the length of the string, instead of the frets and fingers. The slide acts as a string termination and influences the vibration of the string by creating a new load termination in between the nut and the bridge. This allows unique articulations and pitch inflections to be generated as the player is \DIFdelbegin \DIFdel{no-longer }\DIFdelend \DIFaddbegin \DIFadd{no longer }\DIFaddend constrained to the pitches provided by the \DIFdelbegin \DIFdel{fret-locations}\DIFdelend \DIFaddbegin \DIFadd{fret locations}\DIFaddend . Additionally, the interaction of the slide’s surface with that of the string adds a new timbral component related to the slide’s velocity.

In the case of wound strings, this adds two new sounds. The first is a time-varying harmonic component due to the interaction of the slide with the spatially periodic pattern of windings on the string’s surface (inherent in a wound string’s construction).  The second component is due to the stimulation of the string’s longitudinal modes as the slide introduces disturbances in this direction when it impacts the ridges of the windings. As the slide does not provide sufficient force to change the longitudinal length, the longitudinal mode frequencies are static, regardless of the motion of the slide.

Slides \DIFdelbegin \DIFdel{and unwound strings }\DIFdelend are traditionally made \DIFdelbegin \DIFdel{of smooth polished materials, }\DIFdelend \DIFaddbegin \DIFadd{from }\DIFaddend ceramic or metal and \DIFdelbegin \DIFdel{metal respectively. }\DIFdelend \DIFaddbegin \DIFadd{unwound strings are made with metal. These are smooth/polished materials. }\DIFaddend Correspondingly, the coefficient of friction between the \DIFdelbegin \DIFdel{two }\DIFdelend \DIFaddbegin \DIFadd{string and slide }\DIFaddend is comparatively much lower than in the wound-string case. Unwound strings also have a uniform surface, lacking the ridges created by windings which a slide impacts while traveling the length of the string. This darastically reduces the coupling between the slide and the unwound string from a longitudinal standpoint, with the result that the longitudinal modes are not audible. As a result, the contact sound generated for unwound strings is more closely modeled by white-noise scaled by the slide velocity.

\DIFdelbegin \DIFdel{The slide-guitar hasn’t been explored nearly as much from a synthesis standpoint compared to the traditional fretted guitar. Part of my motivation comes from this novelty. Additionally, I have long been a guitar player and slide guitar enthusiast, so understanding the physical mechanisms at play in slide articulation (as well as guitar playing in general) are of keen interest to me personally.
}%DIFDELCMD <

%DIFDELCMD < %%%
\DIFdelend \subsubsection*{Previous Work}
\paragraph{}
Sound synthesis of a vibrating string has long been of interest to researchers as illustrated by the work of Hiller-Ruiz \cite{hiller_synthesizing_1971}\DIFdelbegin \DIFdel{\mbox{%DIFAUXCMD
\cite{hiller_synthesizing_1971-1}}\hskip0pt%DIFAUXCMD
}\DIFdelend , Karplus-Strong \cite{karplus_digital_1983} and Jaffe-Smith \cite{jaffe_extensions_1983}. \DIFdelbegin \DIFdel{Instrument specific extensions were developed by Välimäki et al. \mbox{%DIFAUXCMD
\cite{vaelimaeki_physical_1996}}\hskip0pt%DIFAUXCMD
.  }\DIFdelend Karjalainen et al formalized the more computationally efficient Single Delay Loop (SDL) model \cite{karjalainen_plucked-string_1998}. More recently, guitar specific plucking has been researched by James Woodhouse \DIFdelbegin \DIFdel{in a pair of papers \mbox{%DIFAUXCMD
\cite{woodhouse_synthesis_2004}}\hskip0pt%DIFAUXCMD
\mbox{%DIFAUXCMD
\cite{woodhouse_plucked_2004}}\hskip0pt%DIFAUXCMD
}\DIFdelend \DIFaddbegin \DIFadd{\mbox{%DIFAUXCMD
\cite{woodhouse_synthesis_2004}}\hskip0pt%DIFAUXCMD
}\DIFaddend . None of this work focused on the slide guitar. The first published journal article regarding this topic came from a group of Finnish researchers in 2008 \cite{pakarinen_virtual_2008}. This material is heavily tied to a master’s thesis by Tapio Puputti published in 2010 \cite{puputti_real-time_2012}. This set of papers involves a substantial gestural control component as they aimed to extend a pre-existing Virtual Air Guitar system making the sound synthesis aspect less emphasized. Their approach uses a digital waveguide (DWG) model based on other work from the Helsinki University of Technology.

The same group of researchers has conducted prior work which was incorporated into the slide DWG model. A paper of theirs from 2007 focuses on the interaction noises of fingers with wound strings \cite{pakarinen_analysis_2007}. This heavily informs the model of slide/string surface interaction and translates into a parametric model dependent on the slide and string properties. The same paper influenced a synthesis model of the Chinese guqin \cite{penttinen_model-based_2006}. One of the distinguishing characteristics of the guqin is that it is a fretless plucked instrument which uses wound strings. Correspondingly, the friction model developed for the Virtual Slide \DIFdelbegin \DIFdel{Gutiar }\DIFdelend \DIFaddbegin \DIFadd{Guitar }\DIFaddend (VSG) model is heavily influenced by \DIFdelbegin \DIFdel{the guqin model's}\DIFdelend \DIFaddbegin \DIFadd{that of the guqin}\DIFaddend .

Recent work has attempted to extend the DWG model or use a finite-difference approach. Gianpaolo Evangelista has a history of modeling player/instrument interaction aspects of guitar playing \cite{evangelista_physical_2011} \cite{evangelista_playerinstrument_2010}. His latest approach extendeds the Finnish DWG model to make the model more physically informed \cite{evangelista_physical_2012}. Researchers at Queen’s University Belfast have produced a finite difference model approach \cite{bhanuprakash_finite_2020}.

\subsubsection*{Proposed Research / Methodology}
\paragraph{}
A synthesis model following Puputti's thesis will be developed which supports sample-by-sample calculations of the sounds. The constituent components \DIFdelbegin \DIFdel{(i.e. pulse generator, loop-filter, fractional delay...) }\DIFdelend will be made as modular as possible to allow for an exploration of different approaches to each component and their effect on the sound-synthesis system in general. \DIFdelbegin \DIFdel{This includes a SDL acting as the waveguide, a }\DIFdelend \DIFaddbegin \DIFadd{The components to be explored are: the noise generation mechanism, which forms the basis of the contact sounds; the fractional delay filter, which allows for more precise tuning; the method by which harmonics in the contact sounds are generated; and the coupling of the contact sound generator (CSG) output to the DWG input.
}

\DIFadd{The different noise sources investigated will be decaying noise pulse train as in \mbox{%DIFAUXCMD
\cite{pakarinen_virtual_2008} }\hskip0pt%DIFAUXCMD
and white noise as in the Guqin approach \mbox{%DIFAUXCMD
\cite{penttinen_model-based_2006}}\hskip0pt%DIFAUXCMD
. The different fractional delay methods explored will be }\DIFaddend 5th-order \DIFdelbegin \DIFdel{Lagrange interpolator as the fractional delay element and zero-order energy-preserving interpolation to implement the energy compensation.
}\DIFdelend \DIFaddbegin \DIFadd{Largange interpolation as suggested by Välimäki \mbox{%DIFAUXCMD
\cite{valimaki_discrete-time_1995} }\hskip0pt%DIFAUXCMD
as well as cross faded all-pass interpolation as described by Van Duyne et al. \mbox{%DIFAUXCMD
\cite{duyne_lossless_1997}}\hskip0pt%DIFAUXCMD
. The different approaches to harmonic generation will be the hyperbolic tangent function as described in the original CMJ article \mbox{%DIFAUXCMD
\cite{pakarinen_virtual_2008} }\hskip0pt%DIFAUXCMD
and using a bank of harmonically related resonators as described in the Guqin model \mbox{%DIFAUXCMD
\cite{penttinen_model-based_2006}}\hskip0pt%DIFAUXCMD
. The different CSG to DWG couplings explored will the extremes of complete coupling/decoupling as well as partial coupling via filtering. This will need to be done after the initial tuning of the CSG levels has been completed.
}\DIFaddend

\DIFdelbegin \DIFdel{Physical measurements will be made recreating the original approach to capturing the longitudinal modes.
This will be done using my miniDSP UMIK-2 to capture the sound in }\DIFdelend \DIFaddbegin \DIFadd{First a basic functional verification of each component (delay lines, filters, noise generators...) will be done ensuring their inputs and outputs operate correctly in isolation. Next, the string synthesis tuning and spectrum will be verified by spectral parabolic interpolation to estimate the spectral peaks and ensure the synthesized harmonics are appropriately placed. The CSG will be verified by attempting to synthesize recordings which isolate the interaction noises from \mbox{%DIFAUXCMD
\cite{pakarinen_analysis_2007}}\hskip0pt%DIFAUXCMD
. Objectively these will be compared via spectrogram comparison, while subjectively they will be perceptually evaluated. The entire system will be verified by recording simple slide licks and doing a similar spectral/perceptual comparison to determine the accuracy and usability of }\DIFaddend the \DIFdelbegin \DIFdel{semi-anechoic room available at CIRMMT. The instrument used will be my own Yamaha steel string acoustic guitar as well as my own brass}\DIFdelend \DIFaddbegin \DIFadd{model as well as understand where it is sonically deficient.
}

\DIFadd{Any recordings}\DIFaddend /\DIFaddbegin \DIFadd{measurements will be done a Yahmaha-502 steel string acoustic guitar fitted with medium gauge D'Addario phosphor bronze strings using Dunlop brass/}\DIFaddend chrome/\DIFdelbegin \DIFdel{glass slides. The number of windings on each string and their dimensions will be measured using the micrometer available in the music tech area recreating the analysis for the friction sounds from the guqin paper \mbox{%DIFAUXCMD
\cite{penttinen_model-based_2006}}\hskip0pt%DIFAUXCMD
. This information is used to help calibrate the contact sound generator part of the model. Other measurements will be made as the project progresses to help extend/verify the model. Ultimately the limitations of the synthesis model will dictate subsequent investigations.
Techniques from the more recent approaches, such as in \mbox{%DIFAUXCMD
\cite{bhanuprakash_finite_2020} }\hskip0pt%DIFAUXCMD
or \mbox{%DIFAUXCMD
\cite{evangelista_physical_2012}}\hskip0pt%DIFAUXCMD
, will be investigated to assess their contributions to realism or creative potential (ie.improving the loop filter to incorporate the body effects, adding multiple polarizations...) }\DIFdelend \DIFaddbegin \DIFadd{glass slides. A miniDSP UMIK-2 reference mic will be used to capture the sounds in the semi-anechoic room available at CIRMMT}\DIFaddend .

\DIFdelbegin \DIFdel{After establishing a complete synthesis model, the next step is to determine the effective parameter ranges which produce musical results. The best set of synthesis parameter configurations will be characterized. This could be done in real-time using a GUI, ultimately dependent on the nature of the control signals involved and the best method to control them. Iconic slide licks will serve as the basis of emulation given that the goal is to be able to use the model to create musical sounds .
}%DIFDELCMD <

%DIFDELCMD < %%%
\DIFdelend \subsubsection*{Contributions / Summary}
\paragraph{}
To summarize, a synthesis model of the slide guitar will be developed. This will be based on the DWG model proposed in \cite{pakarinen_virtual_2008} and \cite{puputti_real-time_2012}. After establishing correctness of the initial algorithm, explorations will be made to investigate how different \DIFdelbegin \DIFdel{parametrizations/implementations }\DIFdelend \DIFaddbegin \DIFadd{configurations }\DIFaddend of the constituent components affect the sound. \DIFdelbegin \DIFdel{Physical measurements }\DIFdelend \DIFaddbegin \DIFadd{Sound recordings }\DIFaddend will be made to \DIFdelbegin \DIFdel{examine aspects such as the winding noise and longitudinal mode characteristics. Once the best model implementation has been decided, the parameter space will be explored to see what produces the most usable and musical results}\DIFdelend \DIFaddbegin \DIFadd{aid in verification. Each synthesizer variation will be compared to the others both objectively and subjectively}\DIFaddend .

\clearpage
\bibliography{references}
\bibliographystyle{alpha}

\end{document}